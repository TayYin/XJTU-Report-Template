%---------------------------------------------------------------------
%
%                      Project Name: XJTU Report Templete 
%
%---------------------------------------------------------------------
%
%                 Forked from Qingyun Fang <fqy2017@gmail.com>
%
%
%---------------------------------------------------------------------
%
%                 modified by Lapalaca Yim <lapalacayim@gmail.com>
%
%
%---------------------------------------------------------------------

\documentclass[a4paper,12pt]{report}
\usepackage{ctex}
\usepackage{times}
\usepackage{setspace}
\usepackage{fancyhdr}
\usepackage{graphicx}
\usepackage{wrapfig}
\usepackage{array}  
\usepackage{fontspec,xunicode,xltxtra}
\usepackage{titlesec}
\usepackage{titletoc}
\usepackage[titletoc]{appendix}
\usepackage[top=30mm,bottom=30mm,left=20mm,right=20mm]{geometry}
\usepackage{cite}
\usepackage{listings}
\usepackage[framed,numbered,autolinebreaks,useliterate]{mcode} % 插入代码
\usepackage{minted} %minted渲染代码
\XeTeXlinebreaklocale "zh"
\XeTeXlinebreakskip = 0pt plus 1pt minus 0.1pt

%---------------------------------------------------------------------
%    其他设置
%---------------------------------------------------------------------
%---------------------------------------------------------------------
%	页眉页脚设置
%---------------------------------------------------------------------
\fancypagestyle{plain}{
	\pagestyle{fancy}      %改变章节首页页眉
}

\pagestyle{fancy}
\lhead{\kaishu~第四次作业~}
\rhead{\kaishu~2160100100~二狗子~}
\cfoot{\thepage}

%---------------------------------------------------------------------
%	章节标题设置
%---------------------------------------------------------------------
\titleformat{\chapter}{\centering\zihao{-1}\heiti}{\chinese{chapter}}{1em}{}
\titlespacing{\chapter}{0pt}{*0}{*6}

%---------------------------------------------------------------------
%	摘要标题设置
%---------------------------------------------------------------------
\renewcommand{\abstractname}{\zihao{-3} 摘\quad 要}

%---------------------------------------------------------------------
%	参考文献设置
%---------------------------------------------------------------------
\renewcommand{\bibname}{\zihao{2}{\hspace{\fill}参\hspace{0.5em}考\hspace{0.5em}文\hspace{0.5em}献\hspace{\fill}}}

%---------------------------------------------------------------------
%	引用文献设置为上标
%---------------------------------------------------------------------
\makeatletter
\def\@cite#1#2{\textsuperscript{[{#1\if@tempswa , #2\fi}]}}
\makeatother

%---------------------------------------------------------------------
%	目录页设置
%---------------------------------------------------------------------
\titlecontents{chapter}[0em]{\songti\zihao{-4}}{\thecontentslabel\ }{}
{\hspace{.5em}\titlerule*[4pt]{$\cdot$}\contentspage}
\titlecontents{section}[2em]{\vspace{0.1\baselineskip}\songti\zihao{-4}}{\thecontentslabel\ }{}
{\hspace{.5em}\titlerule*[4pt]{$\cdot$}\contentspage}
\titlecontents{subsection}[4em]{\vspace{0.1\baselineskip}\songti\zihao{-4}}{\thecontentslabel\ }{}
{\hspace{.5em}\titlerule*[4pt]{$\cdot$}\contentspage}

\begin{document}

% 封面
\begin{titlepage}
	\begin{center}
		
    
    
    \textbf{\zihao{0}\kaishu{系统分析与设计作业报告}}\\[0.8cm]
    \textbf{\zihao{2}\kaishu{第四次}}\\[2.3cm]
    \vspace{10mm}
    \includegraphics[width=0.4\textwidth]{figure//singlelogo}\\
    \vspace{10mm}
    
	\vspace{\fill}
	
\setlength{\extrarowheight}{3mm}
{\songti\zihao{2}	
\begin{tabular}{rl}
	
	{\makebox[4\ccwd][s]{班\qquad 级:}}& ~\kaishu \qquad  书法64\\
	
	{\makebox[4\ccwd][s]{姓\qquad 名:}}& ~\kaishu \qquad  二狗子 \\ 

    {\makebox[4\ccwd][s]{学\qquad 号:}}& ~\kaishu \qquad  2160100100 \\ 
    
    {\makebox[4\ccwd][s]{学\qquad 期:}}& ~\kaishu \qquad  2018\textasciitilde 2019第一学期 \\ 
    
    {\makebox[4\ccwd][s]{日\qquad 期:}}& ~\kaishu \qquad  \today \\ 
   

\end{tabular}
 }
	\end{center}	
\end{titlepage}

% 若需要使用摘要,则取消注释
%%---------------------------------------------------------------------
%  摘要页
%---------------------------------------------------------------------
\begin{abstract}
\begin{spacing}{1.5}
	{\zihao{-4}
	南京理工大学(Nanjing University of Science and Technology)是中华人民共和国工业和信息化部直属的一所以工为主,理、工、文、经、管、法、教、艺等多学科协调发展的全国重点大学,是国家“211工程”、“985 工程优势学科创新平台”重点建设高校之一,是“111计划”、“卓越计划”、“中俄工科大学联盟”入选高校之一,素有“兵器技术人才摇篮”的美誉。
	
	1953年,南京理工大学由中国人民解放军军事工程学院(简称“哈军工”)分建而成,先后经历炮兵工程学院、华东工程学院、华东工学院等发展阶段,1993 年更为现名。
	
	学校北依紫金山,西临明城墙,校园占地3118亩。校舍建筑总面积98万平方米,
	南京理工大学是江苏唯一连续四届获得“江苏省十大专利金奖”和“十大专利发明人”称号的单位,并创办有全国第一个依托大学和大学科技园建设的国家专利产业化试点基地,为国防和国民经济建设均做出了重要贡献。\\[0.5cm]
	\textbf{关键字}:\quad 南理工 \quad 电光院 \quad 985平台 \quad211工程
	}
\end{spacing}
\end{abstract}



%---------------------------------------------------------------------
%  目录页
%---------------------------------------------------------------------
\tableofcontents % 生成目录

%---------------------------------------------------------------------
%  作业要求
%---------------------------------------------------------------------
\chapter{作业要求}
\setcounter{page}{1}
\begin{spacing}{1.5}
\songti\zihao{-4}

观察以下代码:

\begin{minted}[frame=lines,
               framesep=2mm]{csharp}
string title = "This is a Unicode π in the sky"
/*
Defined as $\pi=\lim_{n\to\infty}\frac{P_n}{d}$ where $P$ is the perimeter
of an $n$-sided regular polygon circumscribing a
circle of diameter $d$.
*/
const double pi = 3.1415926535
\end{minted}


引用\cite{Leslie.{1994}},如图~\ref{xjtu}所示:
\begin{figure}[hbtp]
  \centering
  \includegraphics[width=0.7\textwidth]{figure/logo}
  \caption{xjtu}
  \label{xjtu}
\end{figure}



\end{spacing}

%---------------------------------------------------------------------
%  系统分析
%---------------------------------------------------------------------
\chapter{系统分析}

\begin{spacing}{1.5}
\section{背景知识}

\end{spacing}

%---------------------------------------------------------------------
%  实验感想
%---------------------------------------------------------------------
\titleformat{\chapter}{\centering\zihao{-1}\heiti}{}{1em}{}
\chapter{实验感想}
\begin{spacing}{1.5}
    没什么好说的。
\end{spacing}

%---------------------------------------------------------------------
%  参考文献设置
%---------------------------------------------------------------------
\addcontentsline{toc}{chapter}{参考文献}

\begin{thebibliography}{99}
\songti \zihao{-4}     
    \bibitem{Leslie.{1994}}
    Leslie Lamport. LATEX: A Document Preparation System.AddisonWesley, Reading, Massachusetts, second edition, 1994, ISBN 0-201-52983-1.
    
\end{thebibliography}

%---------------------------------------------------------------------
%  附录设置
%---------------------------------------------------------------------
\titleformat{\chapter}{\heiti\Large}{附录~\Alph{chapter}}{11pt}{\Large}
\titlespacing{\chapter}{0pt}{*-4}{*4}

\lstset{breaklines}                %自动将长的代码行换行排版
\lstset{extendedchars=false}
\lstset{language=Matlab}
\renewcommand{\thechapter}{附录\Alph{chapter}.} 
\appendix
\begin{appendix}
    
    
\chapter{数据表}
\zihao{-4}\songti
\begin{spacing}{1.5}
    hello world!
\end{spacing}


\end{appendix}
        

\end{document}